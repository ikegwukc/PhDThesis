\chapter{Standard Machine Learning Language Supplemental Code}
\section{Iris Python Code}

This shows the code required to replicate the same actions of the SML \(Query\) in Figure \ref{fig:SML:IrisQuery}. It's important to note that detailed documentation is publicly available in \textsuperscript{\ref{lab:iris:git}}, the purpose of this figure is to highlight the level of complexity relative to a SML query.

\begin{lstlisting}[language=python]
import pandas as pd
import numpy as np

from sklearn.preprocessing import label_binarize
import sklearn.cross_validation as cv
from sklearn.multiclass import OneVsRestClassifier
from sklearn.svm import SVC
from sklearn.metrics import roc_curve, auc

import matplotlib.pyplot as plt
import seaborn as sns
names = ['sepal length(cm)', 'sepal width(cm)', 'petal length(cm)', 'petal width(cm)', 'species']
data = pd.read_csv('../data/iris.csv', names=names)

iris_classes = ['Iris-setosa', 'Iris-versicolor', 'Iris-virginica']
features = np.c_[data.drop('species',1).values]
labels = label_binarize(data['species'], classes=iris_classes)
n_classes = labels.shape[1]

x_train, x_test, y_train, y_test = cv.train_test_split(features, labels, test_size=0.25)
svm = OneVsRestClassifier(SVC(kernel='linear', probability=True))
l = svm.fit(x_train, y_train)
predict_score = model.decision_function(x_test)
test_set_results = model.score(x_test, y_test) * 100
print ('SVM Prediction Accuracy = {0:6.2f}%'.format(test_set_results) )
 fpr = dict()
tpr = dict()
roc_auc = dict()

for i in range(n_classes):
fpr[i], tpr[i], _ = roc_curve(y_test[:, i], predict_score[:, i])
roc_auc[i] = auc(fpr[i], tpr[i])
plt.rcParams['figure.figsize']=(12,12)
# Class Info
columns = [0,1,2,3]
cmap_class = ['Purples_r', 'Greens_r', 'Oranges_r', 'Greys_r' ]
color_class1D = ['purple', 'darkgreen', 'orange', 'grey']
column_headers =  data.columns.values.tolist() # Grab headers from df
column_headers = [column_headers[x] for x in columns] # Map headers to indices selected

label = 'species'
fig, ax = plt.subplots(len(columns), len(columns))
for ic, cc, cc1D in zip(iris_classes, cmap_class, color_class1D): 
  iris_class_data = data.loc[data.species == ic] # sep class
   
  #Generate kde plot matrix for class
  for col1, i in enumerate(columns):
      for col2, j in enumerate(columns):
          if i == j:
              sns.kdeplot(iris_class_data[iris_class_data.columns[col1]], ax=ax[col1][col2], color=cc1D, shade=True, legend=False)
          else:
              sns.kdeplot( iris_class_data[iris_class_data.columns[col1]], iris_class_data[iris_class_data.columns[col2]], ax=ax[col1][col2], cmap=cc)    
          # Formatting
          if j == 0:
              ax[i,j].set_xticklabels([])
              ax[i,j].set_ylabel(column_headers[i])
              ax[i,j].set_xlabel('')
              if i == len(columns)-1:
                  ax[i,j].set_xlabel(column_headers[j])
          elif i == len(columns)-1:
              ax[i,j].tick_params(axis='y', which='major', bottom='off')
              ax[i,j].set_yticklabels([])
              ax[i,j].set_xlabel(column_headers[j])
              ax[i,j].set_ylabel('')                
          else:
              ax[i,j].set_xticklabels([])
              ax[i,j].set_xlabel('')
              
              ax[i,j].set_yticklabels([])
              ax[i,j].set_ylabel('')
  
plt.show()
plt.close()
\end{lstlisting}

\clearpage

\section{Auto-MPG Python Code}
This shows the code required to replicate the same actions of the SML \(Query\) in Figure \ref{fig:SML:AutoMPGQuery}. It's important to note that detailed documentation is publicly available in \textsuperscript{\ref{lab:SML:AUTO}}, the purpose of this figure is to highlight the level of complexity relative to a SML query.
\begin{lstlisting}[language=python]
import pandas as pd
import numpy as np

import matplotlib.pyplot as plt
from sklearn import linear_model
from sklearn.cross_validation import train_test_split
from sklearn.learning_curve import learning_curve, validation_curve

import matplotlib.pyplot as plt
import seaborn as sns

plt.rcParams['figure.figsize']=(12,12)
sns.set()
 
names = ['mpg', 'cylinders', 'displacement', 'horsepower', 'weight', 'acceleration', 'model_year', 'origin', 'car_name']
 
#load dataset
data = pd.read_csv('../data/auto-mpg.csv', sep = '\s+', header = None, names = names)
data_clean=data.applymap(lambda x: np.nan if x == '?' else x).dropna()
X = data_clean[['cylinders', 'displacement', 'horsepower', 'weight', 'acceleration', 'model_year', "origin"]]
#Select target column
y = data_clean['mpg']
#Split data into training and testing sets
X_train, X_test, y_train, y_test = train_test_split(X, y, train_size=0.8, test_size=0.2)

# Define and train  linear regression model
estimator = linear_model.LinearRegression()# Generate Learning Cures
train_sizes, train_scores, test_scores = learning_curve(estimator, X_train, y_train) 
# Train Linear Regression Model
estimator.fit(X_train, y_train)# Generate Validation Curves
param_range = np.arange(0, 5)

v_train_scores, v_test_scores = validation_curve(estimator, X_test, y_test, param_name='normalize', param_range=param_range)

score = estimator.score(X_test, y_test)
print('Accuracy :', score)
g = sns.PairGrid(data_clean, palette='PuOr_r')
g = g.map_diag(sns.kdeplot, shade=True) # can't add color arg...
 
g = g.map_upper(sns.kdeplot, cmap='PuOr_r')
g = g.map_lower(sns.kdeplot, cmap='PuOr_r')
 
plt.show()
plt.close()
 
color_pal = ['purple', 'dark green', 'orange', 'grey'] # For 1-D KDE
cmap_pal = ['PuOr_r'] # For 2-D KDE
classes = [] # May not have a class for categories

column_headers =  data_clean.columns.values.tolist() # Grab headers from df
column_headers = [column_headers[x] for x in columns] # Map headers to indices selected
 
fig, ax = plt.subplots(len(columns), len(columns))
if not classes:
  for col1, i in enumerate(columns):
      for col2, j in enumerate(columns):
          if i == j:
              sns.kdeplot(data_clean[data_clean.columns[col1]], ax=ax[col1][col2], color=color_pal[0], shade=True, legend=False)
          else:
              sns.kdeplot( data_clean[data_clean.columns[col1]], data_clean[data_clean.columns[col2]], ax=ax[col1][col2], cmap=cmap_pal[0])

           # Formatting
           if j == 0:
               ax[i,j].set_xticklabels([])
               ax[i,j].set_ylabel(column_headers[i])
               ax[i,j].set_xlabel('')
                if i == len(columns)-1:
                   ax[i,j].set_xlabel(column_headers[j])
            elif i == len(columns)-1:
                ax[i,j].tick_params(axis='y', which='major', bottom='off')
                ax[i,j].set_yticklabels([])
                ax[i,j].set_xlabel(column_headers[j])
                ax[i,j].set_ylabel('')            
            else:
                ax[i,j].set_xticklabels([])
                ax[i,j].set_xlabel('')            
                ax[i,j].set_yticklabels([])
                ax[i,j].set_ylabel('')
plt.show()
plt.close()
 
plt.figure()
plt.xlabel("Validation examples")
plt.ylabel("Score")
 
v_train_scores_mean = np.mean(v_train_scores, axis=1)
v_train_scores_std = np.std(v_train_scores, axis=1)
v_test_scores_mean = np.mean(v_test_scores, axis=1)
v_test_scores_std = np.std(v_test_scores, axis=1)
 
plt.fill_between(param_range, v_train_scores_mean - v_train_scores_std, v_train_scores_mean + v_train_scores_std, alpha=0.1, color="orange")
plt.fill_between(param_range, v_test_scores_mean - v_test_scores_std, v_test_scores_mean + v_test_scores_std, alpha=0.1, color="purple")plt.plot(param_range, v_train_scores_mean, 'o-', color="orange", label="Training score")
 
plt.plot(param_range, v_test_scores_mean, 'o-', color="purple", label="Cross-validation score")
 
plt.legend(loc="best")
plt.show()
plt.close()
\end{lstlisting}